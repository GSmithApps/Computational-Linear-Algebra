\documentclass{article}
\usepackage[utf8]{inputenc}
\usepackage{amsmath}

\title{Computational Linear Algebra}
\author{Grant Smith}
\date{Spring 2022}

\begin{document}

\maketitle

\section{Matrices}

A table of numbers

\subsection{Matrix Multiplication Properties}

$$ (AB)C = A(BC) = ABC $$

$$ A(B+C) = AB + AC $$

$$ (AB)^{T} = B^{T}A^{T} $$

\subsection{Special Matrices}

\begin{itemize}
    \item Zero Matrix
    \item Identity Matrix
    \item Symmetric Matrices
    $A^T=A$
    
    \item Triagnle matrices or trapezoidal, and diagonal is a subset
    
    
\end{itemize}
\subsubsection{Upper triangular matrix}
\begin{itemize}
 \item easy to solve
\item eigenvalues are right there
\item determinants are right there too   
\end{itemize}

\subsubsection{Lower Triangular}
if they are both upper triangular and lower triangular, then they are diagonal
I don't know much about lower triangular matrices

\subsubsection{Diagonal Matrices}
Very easy to solve a linear system

\section{Essential Problems of Linear Algebra}

\subsection{Solve Linear Systems}

$Ax = b$

Use Gauss Jordan Elimination to solve

Computing determinants is $n!$, so we essentially never do it


\subsection{Eigenvalue Problems}

Eigenvalues are vectors such that

$$Av = \lambda v$$

bridges are essentially eigenvalue problems

\subsection{Singular Value Problem}

This is about finding vectors $u$ and $v$ such that

$$Au = \sigma v$$

Or

$$A^*u = \sigma v$$


\subsection{Matrix Decomposition}
Diagonalization is an example, and SVD is another example.  The goal here is to write a matrix as a product of two other matrices. Here are a couple examples

\subsubsection{LU}

$$A = LU$$

Where $L$ is lower triangular and $U$ is upper triangular

Why is this particular composiiton useful?  All the work is done in computing the two matrices $L$ and $U$, and then once you have that, you can solve everything else cheaply. You're essentially done.

\subsubsection{QR}

This example is $$A = QR$$

Where $Q$ is unitary and $R$ is upper triangular

\subsubsection{SVD}

This one is:

$$A = U\Sigma V^*$$

where $U$ and $V^*$ are unitary and $\Sigma$ is diagonal


\end{document}
